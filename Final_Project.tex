\documentclass[]{article}
\usepackage{lmodern}
\usepackage{amssymb,amsmath}
\usepackage{ifxetex,ifluatex}
\usepackage{fixltx2e} % provides \textsubscript
\ifnum 0\ifxetex 1\fi\ifluatex 1\fi=0 % if pdftex
  \usepackage[T1]{fontenc}
  \usepackage[utf8]{inputenc}
\else % if luatex or xelatex
  \ifxetex
    \usepackage{mathspec}
  \else
    \usepackage{fontspec}
  \fi
  \defaultfontfeatures{Ligatures=TeX,Scale=MatchLowercase}
\fi
% use upquote if available, for straight quotes in verbatim environments
\IfFileExists{upquote.sty}{\usepackage{upquote}}{}
% use microtype if available
\IfFileExists{microtype.sty}{%
\usepackage{microtype}
\UseMicrotypeSet[protrusion]{basicmath} % disable protrusion for tt fonts
}{}
\usepackage[margin=1in]{geometry}
\usepackage{hyperref}
\hypersetup{unicode=true,
            pdftitle={Final Project},
            pdfauthor={Bernard A. Coles IV},
            pdfborder={0 0 0},
            breaklinks=true}
\urlstyle{same}  % don't use monospace font for urls
\usepackage{graphicx,grffile}
\makeatletter
\def\maxwidth{\ifdim\Gin@nat@width>\linewidth\linewidth\else\Gin@nat@width\fi}
\def\maxheight{\ifdim\Gin@nat@height>\textheight\textheight\else\Gin@nat@height\fi}
\makeatother
% Scale images if necessary, so that they will not overflow the page
% margins by default, and it is still possible to overwrite the defaults
% using explicit options in \includegraphics[width, height, ...]{}
\setkeys{Gin}{width=\maxwidth,height=\maxheight,keepaspectratio}
\IfFileExists{parskip.sty}{%
\usepackage{parskip}
}{% else
\setlength{\parindent}{0pt}
\setlength{\parskip}{6pt plus 2pt minus 1pt}
}
\setlength{\emergencystretch}{3em}  % prevent overfull lines
\providecommand{\tightlist}{%
  \setlength{\itemsep}{0pt}\setlength{\parskip}{0pt}}
\setcounter{secnumdepth}{0}
% Redefines (sub)paragraphs to behave more like sections
\ifx\paragraph\undefined\else
\let\oldparagraph\paragraph
\renewcommand{\paragraph}[1]{\oldparagraph{#1}\mbox{}}
\fi
\ifx\subparagraph\undefined\else
\let\oldsubparagraph\subparagraph
\renewcommand{\subparagraph}[1]{\oldsubparagraph{#1}\mbox{}}
\fi

%%% Use protect on footnotes to avoid problems with footnotes in titles
\let\rmarkdownfootnote\footnote%
\def\footnote{\protect\rmarkdownfootnote}

%%% Change title format to be more compact
\usepackage{titling}

% Create subtitle command for use in maketitle
\newcommand{\subtitle}[1]{
  \posttitle{
    \begin{center}\large#1\end{center}
    }
}

\setlength{\droptitle}{-2em}

  \title{Final Project}
    \pretitle{\vspace{\droptitle}\centering\huge}
  \posttitle{\par}
    \author{Bernard A. Coles IV}
    \preauthor{\centering\large\emph}
  \postauthor{\par}
      \predate{\centering\large\emph}
  \postdate{\par}
    \date{3/5/2019}


\begin{document}
\maketitle

\begin{enumerate}
\def\labelenumi{(\arabic{enumi})}
\tightlist
\item
  Research Question
\item
  Not Necessary to deep dive into theory \#\#\#Introduction \#1paragraph
  only The formative years in the lives of many American teenagers take
  place within the context of public or private school systems. In these
  contexts, teens are consistently exposed to many opportunities to
  practice forming and maintaining social relationships. One potential
  outcome of adolescent social relations is the risk of engaging with
  delinquency and violence, especially when teens are unsupervised
  (Haynie \& Osgood 2005). Adolescents remain at increased risk for
  victimization when compared to other age groups. In 1997, 202, 000
  students were victims of nonfatal serious violent crimes at school,
  including rape, sexual assault, robbery, and aggravated assault. When
  adding simple assault to the above, a total of 1.1 million students
  were classified as being victimized in school (Van Dorn 2004). In
  fact, in the United States youth are more than 2.3 times more likely
  than the general population to be victims (Hanish and Guerra 2000).
  Important for this study are instances of serious nonfatal forms of
  victimization and their relationship to the content or behavioral
  norms of the network and the structural characteristics of the
  respondent’s position within the network. School-based risk factors
  for victimization have been studied extensively since data was
  collected for the 1977 Safe School Study conducted by the National
  Institute of Education (Van Dorn 2004). Some often-identified
  correlates of victimization are age and gender, which repeatedly show
  that younger adolescents and males are at the greatest risk of
  physical victimization (Warr, 1993). The prevalence, severity, and
  impacts of this problem have prompted increasing attention in recent
  years by national and international researchers who are committed to
  providing youth with safer schools (Hanish and Guerra 2000). However,
  Van Dorn (2004) found that school-based safety precautions did not
  significantly reduce violent victimization and only showed a trend
  toward significance with nonviolent victimization, suggesting that our
  high schools have much to improve regarding policies geared towards
  preventing teens from engaging in violence on campus. Sociological
  research on adolescent crime and delinquency suggests that the social
  characteristics of an adolescent’s community heavily influence the
  likelihood of that adolescent’s involvement in delinquent behavior
  and perhaps the likelihood of their victimization (Hanish and Guerra
  2000; Schreck, Fisher, and Miller 2004; Mouttapa, Valente, Gallaher,
  Rohrbach and Unger 2004; Sampson 1984; Moody 2001; Berg, Brunson, and
  Stewart 2012; Sampson and Groves 1989; Pridemore 2002; Evans and
  Smokoski 2016; Shaw and Mckay ({[}1942{]} 1969); Hirschi 1969;
  Sutherland and Cressey 1974). Schreck et al. (2004) speculate that
  research on victimization could benefit from studies emphasizing the
  peer influences generated by delinquent groups, because delinquency
  and victimization share many empirical connections. Their study
  identified peer delinquency as a significant risk factor for violent
  victimization. Additionally, Hirschi’s (1969) theory of social
  control and Sutherland’s (1974) theory of differential association
  have both empirically verified a connection between social networks
  and delinquency/crime and have been adapted and replicated by modern
  sociologists such as Haynie (2001), Matsueda (1982), and Mangino
  (2009). For example, Haynie (2001) shows that any structural network
  location that puts one in a position of greater influence within the
  group amplifies an individual’s delinquency above the delinquent
  content of the peer network. Just as there is a relationship between
  network position and delinquency, it follows, that there is a likely
  connection between the social network occupied by an adolescent and
  his/her chances of becoming a victim. However, the likelihood of
  victimization compared to the propensity towards delinquency may
  function differently in terms of causal mechanisms, particularly the
  mediating effects of the network’s structure. I posit that network
  structure influences how an individual may learn appropriate behavior
  and that the specific behaviors they adopt will either enhance or
  limit their exposure to violence.
\end{enumerate}

\subsubsection{Data}\label{data}

\emph{Add Health} To test a friendship network’s influence on violent
victimization, this study employs the public use data from the first
wave of the National Longitudinal Study of Adolescent to Adult Health
(Add Health ). The data consists of a nationally representative sample
of teens, grades 7 – 12, nested in randomly selected public and
private schools throughout the United States in 1994-95. Information on
the sample was collected from the respondents, their peers, school
administrators, parents, siblings, and romantic partners through an
initial in-school survey followed by four in-home interviews.
\emph{In-School Surveys} Add Health’s In-School Questionnaire, a
self-administered instrument, was distributed to more than 90,000
students in grades 7 through 12 in an hour-long class period between
fall 1994 and spring 1995. The questionnaire consisted of many topics,
from education and parental occupation to self-esteem and risk
behaviors, but most important to this study was the information
collected on student’s behaviors and friendships. Respondents were
asked to name their five closest female friends and their five closest
male friends. In instances where the friendship nominations were members
of the same school as the respondent, as more than 80 percent of
nominations were, they too were respondents, meaning that data was also
available on them. Because Add Health project staff assigned an
identification number to each student and recorded these nominations by
each student’s registered ID it is possible to reconstruct the social
networks for most students. This network information makes it possible
to calculate behavioral attributes present in each respondent’s own
friendship (ego) network, such as delinquency, as well as test the
structural influences the network may have on behavior or propensity to
victimization. \emph{In-Home Interviews} Data from the more in-depth
in-home interviews contains sensitive information on the adolescents
such as experience with drugs and alcohol and various other risky
behaviors such as carrying a weapon. One of the most advantageous
components of this in-home method was the use of laptop computers which
played prerecorded questions about experiences with victimization. This
method of data collection helped to maintain confidentiality on numerous
sensitive subjects. These self-reported experiences from the first wave
of in-home interviews was used to construct the dependent variable –
victim violence – for this study. Therefore, the research sample for
this project comes from the in-home wave 1 respondents with the network
data from the in-school survey data as an addition to the sample. The
final research sample for the study consisted of 4172 observations.

\subsubsection{Variable Construction}\label{variable-construction}

\emph{The Dependent Variable: Violent Victimization} The variable victim
violence is a composite indicator of victimization experienced in the
twelve months prior to the wave 1 in-home interviews. It takes on the
value of 0, if the respondent experienced none of the forms of physical
victimization listed in Table 2, or 1 if they have experienced at least
one of the forms. The variable was designed to measure purely physical
manifestations of victimization i.e.~being shot, stabbed or jumped.
Table 1 shows that the victimization variable has a mean of .19 and a
standard deviation of .39 for the research sample. \emph{Popularity,
Centrality, and Standing out} Being a highly visible member, that is
standing out, in a delinquent network is likely to increase one’s
chances of becoming a victim, as theorized above. Two variables have
been chosen to operationalize this concept. Popularity is a measure of
the number of friendship nominations received by the respondent. The
nominations range from 0 to 30 with a mean of 4.81 and a standard
deviation of 3.79. When a person receives more friendship nominations it
is a stark example of high visibility with in an adolescent’s school.
The second operationalization of standing out is centrality. Centrality
is a measure of the number of links required to link all other peers in
the adolescent’s friendship network. Centrally situated adolescents
stand out because they are a focal node, much of the information flowing
through the network flows through members with high centrality scores.
The centrality variable is calculated in the Add Health data using
Bonacich’s formula (Bonacich 1987). Centrality for the research sample
has a mean of .85 and a standard deviation of .62. The variable ranges
from 0 to 4.29. \emph{Density and Blending In} The concept of blending
in is operationalized by the variable density. A highly dense network is
marked by uniformity and a lack of individuality (Bearman, 1991). When
there is less individuality, each member stands out less, that is, they
blend in more. High density thus functions to provide the protective
shell discussed above when coupled with higher rates of peer
delinquency. The variable is defined as the number of ties in the
adolescent’s friendship network divided by the total number of ties
possible. The variable is represented in the research sample as a
percentage and ranges from 7.6 to 100. The mean for the sample is 29.72
with a standard deviation of 14.01. \emph{Weapon Carrying}

\emph{Network Delinquency}

\subsubsection{Methods/Analytic Strategy: Rigorously graded, why you
chose models, why these
priors.}\label{methodsanalytic-strategy-rigorously-graded-why-you-chose-models-why-these-priors.}

The dependent variable for this paper is dichotomous and therefore, the
normal assumptions of ordinary least squared regression cannot be
maintained. To compensate, logistic regression analysis, designed to
handle dependent variables of this nature, was used to analyze the data.
The logistic regression analysis for this project interprets the odds
ratios for the independent variables that represent the individual
variable’s influence on the likelihood of victimization while holding
all other variables in the equation constant. An odds ratio of 1.5 can
be understood as a 50\% increase in the likelihood of violent
victimization for the given variable net of the other variables present
in the equation. On the other hand, an odds ratio of .5 signifies a 50\%
decrease in the likelihood of violent victimization for the variable in
question while controlling for all other variables present in the model.
The statistical method used in this paper anticipates victimization as
the data used to construct the primary independent variables were
collected in the first in-school survey while the data used to construct
the variable victimization come from the in-home interviews conducted
approximately a year after the first wave of in-school surveys were
administered.

\begin{enumerate}
\def\labelenumi{(\arabic{enumi})}
\setcounter{enumi}{5}
\tightlist
\item
  Results: Answer the research question.
\end{enumerate}


\end{document}
